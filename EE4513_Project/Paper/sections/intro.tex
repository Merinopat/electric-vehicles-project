In today's world, the presence of Electric Vehicles (EVs) is increasingly significant, representing a transformative shift in the automotive industry and a critical step toward reducing global carbon emissions and achieving sustainability goals.

This project is not carried out for the purpose of deciding whether it is more sustainable or not to have an EV. Instead, it focuses on the physics behind the efficiency and work performed by Internal Combustion Engines (ICEs) vehicles, EVs and hybrids.

This project seeks to explore and quantify how electric vehicles compare to conventional and hybrid vehicles in terms of efficiency. Rather than addressing the broader question of overall sustainability, the focus lies in the physics of energy conversion and fuel consumption. To evaluate the performance of each drivetrain type under realistic conditions, two standardized drive cycles are used: the WLTC, representing everyday driving behavior, and the US06, which simulates more aggressive acceleration and higher speeds. In addition to differentiating vehicles by engine type (electric, combustion, or hybrid), the analysis also accounts for variations in body type and drivetrain configuration — such as front-wheel drive (FWD), rear-wheel drive (RWD), and all-wheel drive (AWD). By examining how these factors influence vehicle behavior across different scenarios, the project aims to provide a clearer picture of their efficiencies and the key physical principles that drive them.
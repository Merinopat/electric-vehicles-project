The analysis highlights clear distinctions in fuel consumption across different engine, body and drive types, particularly under varying drive cycle conditions. Grouping vehicles by body type allowed for more equitable comparisons by accounting for differences in mass and dynamics.

Hybrid and electric vehicles consistently demonstrate superior efficiency, with none exceeding 1 kWh/km under the WLTC cycle, and maintaining significant advantages even under the more demanding US06 cycle. In contrast, conventional vehicles not only consume more energy on average but also exhibit greater variability, especially in high-load conditions. These findings underscore the fuel efficiency benefits of electrified powertrains and their potential for reducing energy consumption across a wide range of vehicle classes.

Additionally, Figure \ref{fig: final} offers valuable insight into how drivetrain configuration influences vehicle behavior and efficiency. It becomes evident that front-wheel-drive vehicles typically require more energy to operate compared to rear-wheel and especially all-wheel-drive configurations. Vehicles with rear or all-wheel drive appear to hold an advantage in both efficiency and dynamic performance, highlighting the significant role of drivetrain layout in overall energy consumption.

Across all body types and drivetrain configurations, electric vehicles consistently demonstrate superior efficiency compared to conventional vehicles. Their higher energy conversion rates and minimal heat losses—unlike internal combustion engines—grant them a clear advantage. However, their main limitations remain battery storage capacity and charging infrastructure.

Hybrid vehicles emerge as a compelling middle ground. They offer notable efficiency improvements over gasoline and diesel engines while enabling longer driving ranges. It’s important to note, however, that our analysis is based on energy consumption over a single drive cycle. Over extended periods-particularly after the hybrid battery is depleted-fuel consumption may rise to levels comparable to conventional vehicles.

In the WLTC cycle, electric vehicles were between 9.65\% less efficient and 32.54\% more efficient than hybrid vehicles, showing that under calm driving conditions, some hybrids can nearly match or even slightly outperform EVs in isolated cases. In contrast, under the more aggressive US06 drive cycle, the gap widened considerably, with EVs showing up to 44.29\% greater efficiency compared to hybrids. The lowest value was -0.43\%, a very marginal advantage for the hybrid vehicle. This highlights how hybrid performance tends to degrade more under demanding conditions, where electric drivetrains maintain a stronger advantage.

Looking ahead, the key challenge will be improving battery storage and charging solutions for electric vehicles, while continuing to explore and invest in hybrid technologies. Perhaps the optimal solution lies somewhere in between-balancing the strengths of both electric and hybrid systems.